%Je nach dem in welcher Sprache ihr euer Paper schreiben wollt, benutzt bitte entweder den Deutschen-Titel oder den Englischen (einfach aus- bzw. einkommentieren mittels '%')

%Deutsch
\section{Einleitung und Problemhintergrund}

% Englisch
%\section{Introduction}

Ähnliche wie in der ersten Abgabe, gehört hier die Projekt-Idee mit zumindest folgenden Inhalten beschrieben:
\begin{itemize}
	\item\textbf{1. Absatz (Universum / Forschungsfeld): } \\
	Im ersten Absatz soll das Univsersum / Forschungsfeld beschrieben werden, in dem sich das Position Paper befindet. Der Leser / die Leserin soll aus dem "großen Ganzen" an die eigentliche Problemstellung herangeführt werden und wissen, in welchem Themenfeld sich diese befindet. \\
	
	\textbf{\textit{ACHTUNG}}: Es ist in Ordnung, wenn die Studierenden das Universum / Forschungsfeld in mehr als nur einem Absatz herleiten. Es ist nur wichtig, dass folgende Regel eingehalten wird: ein Absatz beteht zumindest aus 3 Sätzen, wobei ein Satz aus maximal 30 Worten besteht.\\
	
	\item\textbf{2. Absatz (Problemstellung im Universum / Forschungsfeld): } \\
	Im zweiten Teil wird dem Leser / der Leserin das Problem, welche im beschriebenen Universum / Forschungsfeld besteht, aufgezeigt bzw. beschrieben. Die Problemstellung gilt dabei als die Grundlage für das Position Paper, die hier beschrieben wird und die Methodik "WIE man es lösen würde" erklärt wird (in Kapitel 3). \\
	
	\textbf{\textit{ACHTUNG}}: Es ist in Ordnung, wenn die Studierenden die Problemstellung in mehr als nur einem Absatz herleiten. Es ist nur wichtig, dass folgende Regel eingehalten wird: ein Absatz beteht zumindest aus 3 Sätzen, wobei ein Satz aus maximal 30 Worten besteht.\\
	
	\item\textbf{3. Absatz (Lösungsansatz, um die Problemstellung zu bearbeiten): } \\
	Im 3. Teil soll eine Idee bzw. ein möglicher Lösungsansatz "angeteasert" werden, mit dem man die beschriebene Problemstellung bearbeiten möchte. Dabei ist es wichtig, dass der beschriebene Lösungsweg realistisch bzw. umsetzbar und auch nachvollziehbar ist. Es soll ein möglicher Weg sein, den man gehen kann, um die Problemstellung zu bearbieten oder sich zumindest einer Lösung anzunähern. Der Leser bzw. die Leserin soll den Eindruck bekommen, dass die Problemstellung tatsächlich damit lösbar sei. \\
	
	\textbf{\textit{ACHTUNG}}: Es ist in Ordnung, wenn die Studierenden den Lösungsansatz in mehr als nur einem Absatz herleiten. Es ist nur wichtig, dass folgende Regel eingehalten wird: ein Absatz beteht zumindest aus 3 Sätzen, wobei ein Satz aus maximal 30 Worten besteht.

	\item\textbf{Letzter Absatz (Aufbau des Position Papers): }\\
	Jedes Paper hat zum Schluss einen Absatz, der beschreibt, wie das vorliegende Position Paper aufgebaut ist. Hier ein Beispiel aus einem Position Paper, das auf Englisch geschrieben wurde: \\
	
	The remainder of this paper is organised as fol- lows: Section II summarises the related work in the field. Next, in Section III, we present the Security Evaluation Framework and explain how it can be used to evaluate the security of SDN-components. Furthermore, we show the general applicability of the proposed framework in an experimental study in Section IV. Finally, in Section V we conclude our work and give an outline of future work in the field.
	
\end{itemize}

\textbf{VORGABE}: dieses Kapitel soll genau eine A4-Seite benötigen