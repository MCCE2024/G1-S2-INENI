%Je nach dem in welcher Sprache ihr euer Paper schreiben wollt, benutzt bitte entweder den Deutschen-Titel oder den Englischen (einfach aus- bzw. einkommentieren mittels '%')

\begin{abstract}
%Deutsch
\textbf{ABSTRACT}
Die Kurzfassung bzw. der Abstract ist ca. so wie ein Kino-Trailer der erste Kontakt mit eurem Paper. Wenn der Trailer nicht gut ist, dann schaue ich mir den Kino-Film erst gar nicht an. Das gleiche gilt auch hier im Abstract, wenn dieser nicht gut ist, wird das Paper gar nicht gelesen. Dabei hat eine gute Kurzfassung / ein guter Abstract 150-300 Wörter und wird in der PAST-Tense geschrieben (nach dem die Arbeit durchgeführt wurde). Darum schreibt man die Kurzfassung / den Abstract auch erst zum Schluss, wenn das Paper bereits fertig geschrieben ist. In jeder Kurzfassung / jedem Abstract sollte man folgende Struktur verfolgen:
\begin{itemize}
	\item\textbf{1-2 Sätze:} Beschreibt in welchem Universum / Forschungsfeld sich das Position Paper befindet
	\item\textbf{1-2 Sätze:} Beschreibt die Problemstellung in diesem Universum / Forschungsfeld
	\item\textbf{1-2 Sätze:} mit welcher Methodik / welchem Forschungsansatz schlägt ihr vor, um die Problemstellung zu bearbeiten?
	\item\textbf{1-2 Sätze:} Welches Ergebnis erwartet man sich bzw. welchen Mehrwert soll der Lösungsvorschlag bringen
	\item\textbf{1-2 Sätze:} Ausblick…was bedeutet das im “Big Picture”
\end{itemize}

%Englisch	
%\textbf{ABSTRACT:}
\end{abstract}