%Je nach dem in welcher Sprache ihr euer Paper schreiben wollt, benutzt bitte entweder den Deutschen-Titel oder den Englischen (einfach aus- bzw. einkommentieren mittels '%')

%Deutsch
%\section{Methodischer Ansatz}

%Englisch
\section{Methodological Approach}

Dieses Kapitel ist das Herzstück eures Position Papers, wo ihr euren Lösungsansatz beschreibt. Je nach dem in welche Richtung euer Thema geht, kann dieses Kapitel auf unterschiedliche Arten aufgezogen werden. Folgende Informationen könnten (müssen jedoch nicht) Teil dieses Kapitels sein:
\begin{itemize}
	\item\textbf{Use Case Beschreibung}\\
	In den meisten Fällen wird man als Startpunkt einen bestimmten Use Case haben, den man als Beispiel heranzieht bzw. den man im Zuge des Projektes bearbeiten möchte / wird. Der Use Case könnte sogar als Einstiegspunkt in die Thematik dienen, wo man die Problemstellung nochmal bildlich darstellt und auf Basis dessen den Lösungsansatz erklärt bzw. motiviert.
	
	Die Use Case Beschreibung soll das \textit{\textbf{"WAS möchte man sich anschauen"}} darstellen und beschreiben.\\
		
	\item\textbf{Technische Beschreibung eures Prototypen / eures Demonstrators}\\
	Wenn man erklärt hat WAS man sich anschauen möchte, wäre die nächste Frage \textit{\textbf{"WIE möchte man es sich anschauen"}}. Dabei wäre eine technische Beschreibung de Veruchsaufbaus / Demonstrators sehr gut geeignet, wo man motiviert, WIE man vor hat die Problemstellung zu bearbeiten. Dieser Teil könnte sogar auf der Use Case Beschreibung aufbauen und die technische Ausführung des Use Cases erklären. 
	
	\item\textbf{Beschreibung was ihr messen / evaluieren / vergleichen / analysieren werdet}\\
	Je nach dem in welchem Themengebiet man unterwegs ist und welche Problemstellung man bearbeitet, wird man unterschiedliche methodische Ansätze verwenden können. Hier geht es vor allem darum zu erklären, welchen Ansatz man wählt um die Problemstellung zu bearbeiten und in welcher Art und Weise man vor hat Daten zu sammeln bzw. Ergebnisse zu erzielen, um das Problem zu bearbeiten / zu lösen. Ein wichtiger Aspekt cabei könnte sein welche Metriken man vor hat zu verwenden, damit man Daten sammeln bzw. Ergebnisse erzielen kann, die zeigen, WIE man die Problemstellung tatsächlich löst.
\end{itemize}

\textbf{VORGABE}: dieses Kapitel soll nicht mehr 3 A4-Seiten benötigen

