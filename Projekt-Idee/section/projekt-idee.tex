% \textbf{\textit{ACHTUNG}}: Es ist in Ordnung, wenn die Studierenden das Universum / Forschungsfeld in mehr als nur einem Absatz herleiten.
% Es ist nur wichtig, dass folgende Regel eingehalten wird: ein Absatz beteht zumindest aus 3 Sätzen, wobei ein Satz aus maximal 30 Worten besteht.

%%\section{Einleitung und Problemhintergrund}
% Bei der ersten Aufgabenstellung sollen die Studierenden ihre Projekt-Idee in Form eines "One Pagers" (genau eine A4-Seite) beschreiben.
% Dabei sollen zumindest folgende Informationen / Inhalte im "One Pager" beschrieben werden

%%\subsection{Forschungsfeld}
% Im ersten Absatz soll das Universum / Forschungsfeld beschrieben werden, in dem sich die Projektarbeit befindet. Der Leser / die Leserin soll aus dem "großen Ganzen" an die eigentliche Problemstellung
% herangeführt werden und wissen, in welchem Themenfeld sich diese befindet.

%%\subsection{Problemstellung}
% Im zweiten Teil wird dem Leser / der Leserin das Problem, welche im beschriebenen Universum / Forschungsfeld besteht, aufgezeigt bzw. beschrieben.
% Die Problemstellung gilt dabei als die Grundlage für die Projekt-Arbeit, die es im Zuge des 2. Semesters zu lösen gilt.
% Die Lösung soll dabei mittels einer technischen Umsetzung (z.B.: einem Prototypen bzw. einer Demo) demonstriert werden.

%%\subsection{Lösungsansatz}
% Im letzten Teil soll eine Idee bzw. ein möglicher Lösungsansatz präsentiert und beschrieben werden, mit dem man die beschriebene Problemstellung bearbeiten möchte.
% Dabei ist es wichtig, dass der beschriebene Lösungsweg realistisch bzw. umsetzbar und auch nachvollziehbar ist.
% Es soll ein möglicher Weg sein, den man gehen kann, um die Problemstellung zu bearbieten oder sich zumindest einer Lösung anzunähern.

\section{Introduction and Problem Background}

\subsection{Research Field}
The ongoing rise in energy prices presents a significant challenge for small and medium-sized enterprises (SMEs) in Styria, Austria. 
Energy efficiency and cost optimization have become critical concerns, particularly for businesses operating on-premise server infrastructures. 
The field of research for this project lies at the intersection of energy monitoring, IoT-based data collection, and cloud-based dashboard. 
With the increasing availability of smart metering technology and real-time power market data, there is an opportunity to leverage modern digital tools to enhance energy management practices. 
By integrating various data sources into a unified system, SMEs can gain detailed insights into the power consumption of their on-premise servers, enabling data-driven decision-making to optimize costs and improve sustainability efforts.

\subsection{Problem Statement}

SMEs often lack transparent and real-time insights into the energy consumption of their server infrastructure, making it difficult to respond dynamically to fluctuating energy prices. 
Traditional power monitoring solutions may be too costly or complex to implement, leaving many businesses without an effective strategy to optimize their energy use. 
The absence of an integrated platform that consolidates smart meter data, power consumption insights from connected devices, and real-time energy market prices leads to inefficiencies and unnecessary expenses. 
Additionally, while individual tools exist to monitor energy usage, there is no comprehensive dashboard that provides SMEs with clear visibility into the specific energy consumption patterns of their on-premise servers. 
The goal of this project is to address this gap by developing an integrated solution that provides actionable insights through real-time visualization, helping businesses make informed decisions about their IT infrastructure energy consumption.

\subsection{Solution Approach}
For this work, energy data from the smart meter, as well as from at least one server, will be collected and subsequently sent to the AWS Cloud via an interface.
For data collection from the smart meter, a product from Energie Steiermark called "Energylive" will be used.
In the AWS Cloud, the data will be aggregated using a time series database and visualized in a dashboard.
