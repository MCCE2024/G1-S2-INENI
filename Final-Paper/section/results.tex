%Je nach dem in welcher Sprache ihr euer Paper schreiben wollt,
%benutzt bitte entweder den Deutschen-Titel oder den Englischen (einfach aus- bzw. 
%einkommentieren mittels '%')

%Deutsch
%\section{Ergebnisse}

%Englisch
\section{Results}
\label{results:results}
This section presents the key findings from our energy-aware server management system evaluation, focusing on system implementation outcomes and quantitative energy consumption analysis.

\subsection{System Implementation}
\label{results:system-implementation}
The prototype system was successfully deployed on AWS infrastructure using Infrastructure as Code principles. Key technical achievements include:
\begin{itemize}
    \item Real-time data ingestion with sub-second latency from NOUS A5T smart plugs via MQTT
    \item Reliable 15-minute interval data collection from energyLive API and EPEX Spot API
    \item Scalable DynamoDB storage with consistent sub-10ms query response times
\end{itemize}

\subsection{Energy Consumption Analysis}
\label{results:energy-analysis}
Our analysis revealed distinct power consumption patterns across different workload scenarios, summarized in Table~\ref{tab:workload-comparison}.

\begin{table}[h]
\caption{Workload Power Consumption Comparison}
\label{tab:workload-comparison}
\begin{tabular}{lcccc}
\hline
\textbf{Workload} & \textbf{Avg Power (W)} & \textbf{Peak (W)} & \textbf{Variation (W)} & \textbf{Energy (kWh)} \\
\hline
CPU Stress & 662.25 & 684.0 & ±3.94 & 0.662 \\
I/O Stress & 317.86 & 403.0 & ±52.15 & 0.159 \\
System Reboot & 294.27 & 424.0 & ±71.68 & 0.025 \\
Maintenance & 205.37 & 293.0 & ±42.59 & 0.017 \\
Idle State & 172.20 & 266.0 & ±6.82 & 0.172 \\
\hline
\end{tabular}
\end{table}

Key observations from the workload analysis:
\begin{itemize}
    \item CPU-intensive operations showed the highest power increase (284.6\% above baseline)
    \item I/O operations demonstrated moderate power increase (84.6\% above baseline)
    \item System operations exhibited high variability but lower average consumption
    \item Idle state maintained stable power consumption with minimal variation
\end{itemize}

\subsection{Cost Optimization Potential}
\label{results:optimization}
Analysis of EPEX spot prices during the test period revealed significant cost optimization opportunities:

\begin{itemize}
    \item \textbf{Price Variation:} Daily electricity prices ranged from -1.685 to 14.168 cent/kWh
    \item \textbf{Scheduling Impact:} Optimal workload scheduling could achieve up to 54.6\% cost reduction
    \item \textbf{Annual Savings:} Projected savings for a single server:
    \begin{itemize}
        \item CPU-intensive workloads: 543.85 EUR/year
        \item Maintenance operations: 6.23 EUR/year
        \item Total optimization potential: 550.08 EUR/year
    \end{itemize}
\end{itemize}

\begin{table}[h]
\caption{Cost Comparison Across Scheduling Scenarios}
\label{tab:cost-comparison}
\begin{tabular}{lllll}
\hline
\textbf{Workload} & \textbf{Energy (kWh)} & \textbf{Best Case (¢)} & \textbf{Worst Case (¢)} & \textbf{Savings (\%)} \\
\hline
CPU Stress & 0.662 & 4.10 & 9.04 & 54.6\% \\
I/O Stress & 0.159 & 0.98 & 2.17 & 54.6\% \\
Maintenance & 0.017 & 0.11 & 0.23 & 54.6\% \\
Idle (1h) & 0.172 & 1.07 & 2.35 & 54.6\% \\
\hline
\multicolumn{5}{l}{\small Best Case: 6.198 cent/kWh, Worst Case: 13.648 cent/kWh} \\
\end{tabular}
\end{table}

These results demonstrate that intelligent workload scheduling based on real-time electricity pricing can lead to substantial cost savings for SMEs operating on-premise server infrastructure. The combination of comprehensive energy monitoring and price-aware scheduling provides a practical approach to optimizing operational costs while maintaining service quality.

\subsection{Implementation Considerations and Limitations}
\label{results:limitations}
The analysis revealed several key considerations for implementing energy-aware server management:

\begin{itemize}
    \item \textbf{Workload Constraints:}
    \begin{itemize}
        \item Critical operations cannot always be scheduled optimally
        \item Some workloads require immediate execution regardless of energy costs
        \item Batch processing jobs offer the most flexibility for optimization
    \end{itemize}
    
    \item \textbf{Economic Factors:}
    \begin{itemize}
        \item Price volatility (230.6\% daily variation) enables significant optimization
        \item Benefits scale linearly with server count in SME deployments
        \item Austrian market prices may not reflect other regional patterns
    \end{itemize}
    
    \item \textbf{Technical Limitations:}
    \begin{itemize}
        \item Single server measurements may not represent diverse hardware configurations
        \item Virtualization overhead affects absolute power measurements
        \item Limited test duration (24 hours) may miss longer-term patterns
    \end{itemize}
\end{itemize}

These results demonstrate that intelligent workload scheduling based on real-time energy pricing can provide substantial cost savings for SME server operations, with potential annual savings of 550.08 EUR per server under optimal conditions. The approach is particularly valuable for organizations with flexible workload scheduling and multiple servers, where benefits can be multiplied across the infrastructure. 