%Je nach dem in welcher Sprache ihr euer Paper schreiben wollt, benutzt bitte entweder den Deutschen-Titel oder den Englischen (einfach aus- bzw. einkommentieren mittels '%')

%Deutsch
%\section{Schlussfolgerungen und Ausblick}


%Englisch
\section{Conclusion and Future Work}
Rising energy costs create significant challenges for small and medium-sized enterprises operating on-premise server infrastructures. This final paper addresses the critical gap in energy management solutions specifically tailored for SMEs who lack transparent insights into their server infrastructure's power consumption patterns, as identified in Section~\ref{related_work:related-work}.

The implemented architecture described in Section~\ref{methodology:methodological-approach} leverages cloud-based technologies to create an integrated energy monitoring system that collects real-time consumption data from both smart meters and connected server equipment. The system simultaneously incorporates dynamic electricity pricing information to provide context for consumption patterns. The serverless implementation on AWS provides a scalable, cost-effective solution that minimizes operational overhead, which is critical for resource-constrained SMEs.

The evaluation approach outlined in Section~\ref{methodology:evaluation-approach} successfully profiled energy consumption across various operational scenarios. The results presented in Section~\ref{results:results} demonstrate that energy-aware server management delivers measurable value for SME environments through intelligent workload scheduling based on real-time energy and pricing data. The system achieved reliable data collection with sub-second latency and maintained 99.9\% uptime during evaluation.

This work makes three key contributions to energy-aware computing research. First, it presents the first integrated architecture combining IoT-based server monitoring, smart meter integration, and real-time electricity pricing specifically designed for SME environments. Second, it demonstrates the technical feasibility of cloud-native energy management systems using serverless technologies suitable for resource-constrained organizations. Third, it provides empirical evidence of potential cost savings through intelligent workload scheduling, with preliminary analysis indicating 15-25\% reduction in energy costs for time-shiftable workloads.

Several limitations should be considered when interpreting these results. The evaluation was conducted on a single server configuration, which may not represent the diversity of SME server environments. The measurement period was limited to specific operational scenarios, and the focus on Austrian electricity markets may limit generalizability to other regions with different market structures. Additionally, the current implementation requires manual intervention for workload scheduling decisions.

Future research should address these limitations and explore several promising directions. Machine learning models could predict energy consumption patterns and automate optimal scheduling decisions based on historical data and weather forecasts. The architecture could be extended to support multi-server environments with load balancing considerations. Integration with local renewable energy generation data would enable optimization based on green energy availability. Finally, blockchain-based energy trading mechanisms could allow SMEs to participate in peer-to-peer energy markets for additional cost optimization.

The findings establish that energy-aware server management provides tangible benefits for SME environments. The combination of real-time monitoring, market price integration, and cloud-based analytics creates a practical foundation for reducing energy costs while maintaining operational efficiency. This work successfully addresses the central research question posed in Section~\ref{introduction:introduction}: how much energy each server activity consumes and when is the most cost-effective time to perform it. As energy prices continue to rise and sustainability becomes increasingly important, such integrated approaches will become essential for SME competitiveness and environmental responsibility.