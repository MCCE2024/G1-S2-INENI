%Je nach dem in welcher Sprache ihr euer Paper schreiben wollt, benutzt bitte entweder den Deutschen-Titel oder den Englischen (einfach aus- bzw. einkommentieren mittels '%')

%Deutsch
%\section{Schlussfolgerungen und Ausblick}


%Englisch
\section{Conclusion and Future Work}
The rising energy costs present a significant challenge for small and medium-sized enterprises (SMEs)
operating on-premise server infrastructures. This position paper addresses the critical gap in
energy management solutions specifically tailored for SMEs, who often lack transparent insights
into their server infrastructure's power consumption patterns and struggle to respond dynamically
to fluctuating energy prices. While existing research has made significant advancements in
process-level energy measurement, data center management, and IoT-based building management systems,
our analysis reveals that no comprehensive solution exists that simultaneously delivers
per-process energy attribution, smart-meter integration, real-time electricity price correlation,
and a unified dashboard for actionable insights.

Our proposed architecture leverages cloud-based technologies to create an integrated energy
monitoring system that collects real-time consumption data from both smart meters and connected
server equipment, while simultaneously incorporating dynamic electricity pricing information.
The serverless implementation on AWS provides a scalable, cost-effective solution that minimizes
operational overhead—a critical consideration for resource-constrained SMEs. Through the
combination of IoT devices (NETIO PowerCable), facility-level monitoring via the Energy Live API
(connected to the Sagemcom Smart Meter), and market data integration (EPEX Spot API), our system
offers unprecedented visibility into energy consumption patterns and their economic implications.

The proposed measurement methodology outlines an approach for detailed energy profiling across
various operational scenarios, including maximum computational load, I/O stress testing, system
reboot cycles, maintenance operations, and idle states. This comprehensive approach to energy
measurement would provide the foundation for data-driven decision-making regarding workload
scheduling and infrastructure optimization. By understanding the energy intensity of different
server activities and correlating these with real-time electricity pricing, SMEs could implement
targeted strategies to reduce costs without compromising operational requirements. Our position
paper proposes several important potential benefits for SMEs:

\begin{enumerate}
	\item Server workload scheduling based on real-time energy pricing can yield significant cost
	savings, particularly for energy-intensive computational tasks that can be time-shifted to
	lower-cost periods.
	\item The integration of IoT-based monitoring with cloud analytics provides a scalable solution
	that grows with business needs without requiring substantial upfront investment.
	\item The AWS serverless architecture offers an optimal balance of functionality, security,
	and cost-effectiveness for SME implementations.
	\item Granular visibility into server power consumption enables more informed infrastructure
	planning and optimization.
\end{enumerate}

In the final paper, readers will find five concise, reproducible test scenarios—each emulating a
typical VM action on a bare-metal server—to profile energy consumption and identify optimization
opportunities. These include peak CPU stress, storage and network I/O workloads, full system
reboot cycles, routine maintenance (OS patching), and idle-state baselines. For each scenario,
we detail the tools, execution parameters, monitoring approach, and key metrics, enabling SMEs
to replicate the methodology and correlate energy use with operational states.