%Je nach dem in welcher Sprache ihr euer Paper schreiben wollt, benutzt bitte entweder den Deutschen-Titel oder den Englischen (einfach aus- bzw. einkommentieren mittels '%')

%Deutsch
%\section{Stand des Wissens}

% Englisch
\section{Related Work}
\label{related_work:related-work}
A growing body of research addresses the challenges of energy measurement, management, and optimization in server and data center environments. This section reviews key contributions in four thematic areas: (1) process-level energy measurement, (2) data center management approaches, (3) IoT and building management systems, and (4) policy context and foundations. We conclude with a synthesis of the research gap and the unique contribution of our work.
\paragraph{\textbf{Process-Level Energy Measurement:}}
Recent advances have enabled more granular attribution of energy consumption within server environments. Qiao et al. (2023) introduced Wattmeter, a Linux-level tool that attributes energy usage to individual processes and informs CPU scheduling decisions. While this approach is valuable for operating system research, it does not integrate electricity pricing or provide visualization capabilities, limiting its applicability for cost-aware management \cite{qiao2023wattmeter}. Similarly, Zhang et al. (2022) developed a method for container-level energy attribution in microservice architectures, but did not address real-time market price integration. Khan and Varghese (2021) proposed energy-aware scheduling algorithms for heterogeneous server clusters, focusing on power consumption patterns without considering dynamic electricity pricing \cite{zhang2022container,khan2021energy}.
\paragraph{\textbf{Data Center Management Approaches:}}
Energy optimization in data centers has been approached from multiple perspectives. Chen et al. (2022) applied machine learning to enable cost-aware demand response, achieving significant peak-shaving in large-scale cloud environments. Cheung et al. (2021) proposed linear models correlating CPU utilization with power consumption, offering a simpler alternative for data center operators. Smith et al. (2023) analyzed the trade-offs between cost and carbon emissions, highlighting the complexity of multi-objective optimization. Mahmoud et al. (2020) combined workload prediction with renewable energy forecasts for hybrid optimization, while Wang et al. (2023) employed reinforcement learning to dynamically allocate resources in response to variable electricity prices \cite{chen2022datacentermodel,cheung2021utilizationmodel,smith2023warofefficiencies,mahmoud2020cost,wang2023reinforcement}.
\paragraph{\textbf{IoT and Building Management Systems:}}
The integration of IoT technologies has expanded the scope of energy management beyond traditional data centers. Gholami et al. (2020) provided a comprehensive survey of smart building energy management systems, emphasizing IoT architectures and analytics. Ristić et al. (2021) designed an IoT-based energy management platform specifically for SMEs, demonstrating the potential for tailored solutions in smaller organizations. Palacios-Garcia et al. (2022) explored edge computing for real-time energy management in distributed environments, while Kumar et al. (2021) introduced blockchain-enabled frameworks for secure energy data sharing. Lopez-Garcia et al. (2023) proposed standardized APIs to enhance interoperability among energy management systems \cite{gholami2020energymanagement,ristic2021iotenergymanagement,palacios2022edge,kumar2021blockchain,lopez2023standardized}.
\paragraph{\textbf{Policy Context and Foundations:}}
Policy and regulatory frameworks play a critical role in shaping energy management practices. The International Energy Agency (2023) provides high-level policy guidance for data centers, and the European Commission (2022) has compiled case studies highlighting successful energy efficiency initiatives in SMEs. Siano (2014) established foundational concepts for demand-response programs, while Ahmad et al. (2021) analyzed regulatory frameworks affecting SME energy management across jurisdictions. The Green Grid Association (2021) introduced advanced metrics for assessing data center sustainability, and Koomey and Masanet (2021) provided updated global estimates of data center energy consumption, underscoring the importance of server-level optimizations \cite{iea2023datacenters,ec2022energyefficiencysmes,siano2014demandresponse,ahmad2021regulatory,greengrid2021beyond,koomey2021does}.
\paragraph{\textbf{Gap Analysis and Contribution:}}
Despite significant progress in each of these domains, existing solutions typically address only isolated aspects of the broader challenge. To the best of our knowledge, no prior work offers a holistic system that combines per-process energy attribution in SME servers, smart meter data integration, real-time electricity price correlation, and a unified dashboard for actionable insights. Our work addresses this gap by proposing an AWS-based architecture (Section~\ref{methodology:prototype}) that unifies kernel-level process metering, smart meter feeds (Appendix~\ref{appendix:energylive-api}), and market price polling (Appendix~\ref{appendix:strompreis-api}) into a comprehensive visualization and decision-support platform. The complete implementation details and source code are available in our public repository (Appendix~\ref{appendix:github-docs}), enabling reproducibility and further research in this domain.